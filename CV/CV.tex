% LaTeX Curriculum Vitae Template
%
% Copyright (C) 2004-2009 Jason Blevins <jrblevin@sdf.lonestar.org>
% http://jblevins.org/projects/cv-template/
%
% You may use use this document as a template to create your own CV
% and you may redistribute the source code freely. No attribution is
% required in any resulting documents. I do ask that you please leave
% this notice and the above URL in the source code if you choose to
% redistribute this file.

\documentclass[letterpaper]{article}

\usepackage{hyperref}
\usepackage{geometry}
\usepackage{marvosym} % For cool symbols.

% Comment the following lines to use the default Computer Modern font
% instead of the Palatino font provided by the mathpazo package.
% Remove the 'osf' bit if you don't like the old style figures.
\usepackage[T1]{fontenc}
\usepackage[sc,osf]{mathpazo}

% Set your name here
\def\name{Jihyeug Jang}

% Replace this with a link to your CV if you like, or set it empty
% (as in \def\footerlink{}) to remove the link in the footer:
\def\footerlink{http://jblevins.org/projects/cv-template/}

% The following metadata will show up in the PDF properties
\hypersetup{
  colorlinks = true,
  pdfauthor = {\name},
  pdfkeywords = {mathematics},
  pdftitle = {\name: Curriculum Vitae},
  pdfsubject = {Curriculum Vitae},
  pdfpagemode = UseNone
}

\geometry{
  body={6.5in, 8.5in},
  left=1.0in,
  top=1.25in
}

% Customize page headers
\pagestyle{myheadings}
\markright{\name}
\thispagestyle{empty}

% Custom section fonts
\usepackage{sectsty}
\sectionfont{\rmfamily\mdseries\Large}
\subsectionfont{\rmfamily\mdseries\itshape\large}

% Other possible font commands include:
% \ttfamily for teletype,
% \sffamily for sans serif,
% \bfseries for bold,
% \scshape for small caps,
% \normalsize, \large, \Large, \LARGE sizes.

% Don't indent paragraphs.
\setlength\parindent{0em}

% Make lists without bullets
% \renewenvironment{itemize}{
%   \begin{list}{}{
%     \setlength{\leftmargin}{1.5em}
%   }
% }{
%   \end{list}
% }


% DO NOT DELETE THIS COMMENT!!! MACROS BELOW:


\begin{document}

% Place name at left
{\huge \name}

% Alternatively, print name centered and bold:
%\centerline{\huge \bf \name}

\vspace{0.2in}

\noindent\rule{\textwidth}{1pt}

\section*{Contact Information}
\begin{minipage}{0.45\linewidth}
  University of Geneva \\
  Section de mathématiques \\
  4-04 \\
  Rue du Conseil-Général 7/9 \\
  1205 Genève, Switzerland
\end{minipage}
\begin{minipage}{0.45\linewidth}
  \begin{tabular}{ll}
    Phone: & (+82) 1035448093 \\
    Email: & \href{mailto:jihyeugjang@gmail.com}{\tt jihyeugjang@gmail.com} \\
    Homepage: & \href{https://jihyeugjang.github.io}{\tt https://jihyeugjang.github.io} \\
  \end{tabular}
\end{minipage}


\section*{Research interest}

Enumerative combinatorics and Algebraic combinatorics.

\section*{Employment}
\begin{itemize}
\item Post-doc, University of Geneva, Jan. 2025 - present
    \begin{itemize}
    \item Mentor: \href{https://www.unige.ch/~doussej/}{Jehanne Dousse}
    \end{itemize}
\item Post-doc, Sungkyunkwan University, Mar. 2024 - Dec. 2024
    \begin{itemize}
    \item Mentor: \href{https://jangsookim.github.io/}{Jang Soo Kim}
    \end{itemize}
\end{itemize}


\section*{Education}

\begin{itemize}
  \item Ph.D. in Mathematics, Sungkyunkwan University, February 2024.
    \begin{itemize}
    \item Advisor: \href{https://jangsookim.github.io/}{Jang Soo Kim}
    \end{itemize}

  \item B.S. in Mathematics, Sungkyunkwan University, February 2017.
\end{itemize}

% \section*{Experience}


% \section*{Honors and awards}

% \section*{Editorships}

% \section*{Grants}




\section*{Publications and preprints}


\subsection*{Submitted:}
\begin{enumerate}
\item (with Jang Soo Kim, Jianping Pan, Joseph Pappe, Anne Schilling) Hook-valued tableaux uncrowding and tableau switching
\item (with Minho Song) Combinatorics of orthogonal polynomials on the unit circle
\item (with Jehanne Dousse, Fr\'ed\'eric Jouhet) Andrews--Gordon and Stanton type identities: bijective and Bailey lemma approaches 
\end{enumerate}


\subsection*{Published:}
\begin{enumerate}
\item (with Byung-Hak Hwang, Jang Soo Kim, Minho Song, U-keun Song) Refined canonical stable Grothendieck polynomials and their duals, Part 2, {\it European Journal of Combinatorics}, Volume 127, (2025)
\item (with Louis W. Shapiro, Minho Song) Combinatorial Reciprocity for Riordan Arrays, {\it Linear Algebra and its Applications}, (2025)
\item (with Mark Kempton, Sooyeong Kim, Adam Knudson, Neal Madras, Minho Song) Kemeny’s constant and enumerating Braess edges in trees, {\it Linear and Multilinear Algebra}, 1-37, (2024)
\item (with Byung-Hak Hwang, Jang Soo Kim, Minho Song, U-keun Song) Refined canonical stable Grothendieck polynomials and their duals, Part 1,
  {\it Advances in Mathematics}, Volume 446, (2024)
\item (with Byung-Hak Hwang, Jaeseong Oh) A combinatorial model for the transition matrix between the Specht and web bases,
  {\it Forum of Mathematics, Sigma}, Volume 11, (2023), e82
\item (with Sooyeong Kim, Minho Song) Kemeny’s constant and Wiener index on trees,
  {\it Linear Algebra and its Applications}, Volume 674, (2023), Pages 230-243
\item (with Donghyun Kim, Jang Soo Kim, Minho Song, U-keun Song) Negative moments of orthogonal polynomials,
  {\it Forum of Mathematics, Sigma}, Volume 11, (2023), e22
\item (with Jang Soo Kim) Volumes of flow polytopes related to caracol graphs,
  {\it Electronic J. Combin.}, Volume 27, Issue 4 (2020), P4.21
\end{enumerate}


\section*{Talks and posters}

\begin{enumerate}
\item Andrews--Gordon and Stanton type identities: bijective and Bailey lemma approaches, \href{https://math.skku.edu/math/community/seminar.do?mode=view&articleNo=204004&article.offset=0&articleLimit=10}{Special seminar}, Sungkyunkwan University, Korea, Aug 1, (2025)
\item A hidden symmetry of refined canonical stable Grothendieck polynomials (poster), \href{https://www.math.sci.hokudai.ac.jp/sympo/fpsac2025/}{FPSAC 2025}, Hokkaido Univerisy, Sapporo, Japan, Jul 21-25, (2025)
\item Combinatorics of the orthogonal polynomials on the unit circle, \href{http://events.kias.re.kr/h/CombinatoricsProbability/?pageNo=5480}{Workshop on Combinatorics and Probability}, Korea, Jun 27-28, (2024)
\item Combinatorial reciprocity for Riordan arrays, \href{https://sites.google.com/view/csyr/1st?authuser=0}{1st Combinatorics Seminar for Young Researchers}, Inha University, Korea, Jun 19, (2024)
\item Combinatorial reciprocity for Riordan arrays, \href{https://www.riordanarray.org/}{9th International Symposium on Riordan Arrays and Related Topics}, Howard University, USA, Jun 3-5, (2024)
\item Lattice on permutation tableaux, \href{https://sites.google.com/ajou.ac.kr/tcseminar/}{Topology and Combinatorics seminar at Ajou University}, Online, May 16, (2024)
\item A trim lattice on permutation tableaux, \href{https://www.kms.or.kr/conference/2024_spring/}{2024 KMS Spring Meeting}, Korea, Apr 18-20, (2024)
\item Refined canonical stable Grothendieck polynomials and their duals (poster), \href{http://fpsac23.math.ucdavis.edu/}{FPSAC 2023}, UC Davis, California, USA, Jul 17-21, (2023)
\item Volumes of flow polytopes related to caracol graphs, \href{https://dgeco.math.cnrs.fr/}{Séminaire DGeCo}, Sorbonne Université, France, Apr 18, (2023)
\item Negative moments of orthogonal polynomials, \href{https://lipn.fr/~banderier/Seminaires/}{Journée-séminaire de combinatoire}, Université Paris 13, France, Apr 11, (2023)
\item Negative moments of orthogonal polynomials (poster), \href{https://events.unibo.it/combinatorics}{89th Séminaire Lotharingien de Combinatoire and Brenti Fest}, Centro Residenziale Universitario di Bertinoro, Italy, Mar 26-29, (2023)
\item On sequences related to the pallet loading problem, \href{https://swb.skku.edu/aorc/seminar.do?mode=view&articleNo=39837&article.offset=0&articleLimit=10}{AORC Monthly Seminar }, Online, Jan 27, (2023)
\item On sequences related to the pallet loading problem, \href{http://events.kias.re.kr/h/combinatorics/?pageNo=4839}{The 26th KIAS Workshop on Combinatorics}, Shilla Stay Haeundae, Korea, Dec 20-22, (2022)
\item A combinatorial model for the transition matrix between the Specht and web bases, \href{https://tscrim.github.io/pacs_en.html}{Physical Algebra and Combinatorics Seminar}, Online, Aug 12, (2022)
\item A combinatorial model for the transition matrix between the Specht and web bases (poster), \href{https://math.iisc.ac.in/fpsac2022/}{FPSAC 2022}, Indian Institute of Science, Bangalore, India, Jul 18-22 (2022)
\item A combinatorial model for the transition matrix between the Specht and web bases, \href{https://www.kias.re.kr/kias/activities/seminars/list.do?menuNo=404003&eventTy=&schoolsCd=C&centrspgmsCd=&pageIndex=1&sdate=2021-12-15&edate=2021-12-17&mjrcdnm=&searchCnd=1&searchWord=}{One-day workshop on web bases}, Online, Dec 16, (2021)
\item Refined canonical stable Grothendieck polynomials and their duals, \href{http://www.kkms.org/main/main}{2021 Annual Meeting on the Kangwon-Kyungki Mathematical Society}, Korea, Jul 16, (2021)
\item Volumes of flow polytopes related to the caracol graphs, \href{https://2021.canadam.math.ca/}{CanaDAM 2021 – Online Meeting}, Online, May 25-28, (2021)
\item A permutation interpretation of the transition matrix between the polytabloid and web bases, \href{https://www.kms.or.kr/meetings/spring2021/}{2021 KMS Spring Meeting}, Online, Apr 29-30, (2021)
\item Computing volumes of flow polytopes using labeled Dyck paths, \href{https://cw2019.combinatorics.kr/information}{2019 Combinatorics Workshop}, Songdo, Incheon, Korea, Aug 13-15, (2019) 
\item Computing volumes of flow polytopes using labeled Dyck paths, \href{http://www.kkms.org/main/main}{2019 Annual Meeting on the Kangwon-Kyungki Mathematical Society}, Daegu, Korea, Jun 28-30, (2019) 
\item Combinatorial proof of two constant term identities, \href{http://math.shinshu-u.ac.jp/~nu/html/workshop/20190115-shinshu/}{Workshop on Algebraic and Enumerative Combinatorics}, Shinshu University, Japan, Jan 15-17, (2019)
\end{enumerate}

\section*{Program Languages}
\begin{itemize}
\item \href{https://www.sagemath.org/}{SageMath}
\end{itemize}


% \subsection*{Proceedings}

% \begin{itemize}
% \item A generalized T-Test and measure of multivariate dispersion,
%   Proc. Second Berkeley Symposium of Mathematical Statistics and
%   Probability, 1951.
% \end{itemize}

% \bigskip

% Footer

% \section*{Refrees}

% \begin{tabular}{lr}
% % referee 1
% \begin{minipage}[t]{2.5in}
% prof. \textbf{Jang Soo Kim}\\ \\
% Department of Mathematics,\\
% Sungkyunkwan University\\ \\
% 2066 Seobu-ro, Jangan-gu\\ 
% Suwon, Gyeonggi-do\\ 
% 16419, South Korea\\ \\
%  \href{mailto:jangsookim@skku.edu}{jangsookim@skku.edu}
% \end{minipage}
% &
% % referee 2
% \begin{minipage}[t]{2.5in}
% prof. \textbf{Sylvie Corteel}\\ \\ 
% Department of Mathematic,\\
% University of California Berkeley\\ \\
% Evans Hall, room 933, Berkeley, CA\\
% 94720, USA\\ \\ 
% \href{mailto:corteel@berkeley.edu}{corteel@berkeley.edu}
% \end{minipage}
% \\
% \\ % additional newline for spacing.
% \end{tabular}

\end{document}